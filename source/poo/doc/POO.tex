\documentclass[11pt]{article}

    %\usepackage[breakable]{tcolorbox}
    \usepackage[most]{tcolorbox}
    \usepackage{lmodern} 
    \usepackage{parskip} % Stop auto-indenting (to mimic markdown behaviour)
    
    \usepackage{iftex}
    \ifPDFTeX
    	\usepackage[T1]{fontenc}
    	\usepackage{mathpazo}
    \else
    	\usepackage{fontspec}
    \fi

    % Basic figure setup, for now with no caption control since it's done
    % automatically by Pandoc (which extracts ![](path) syntax from Markdown).
    \usepackage{graphicx}
    % Maintain compatibility with old templates. Remove in nbconvert 6.0
    \let\Oldincludegraphics\includegraphics
    % Ensure that by default, figures have no caption (until we provide a
    % proper Figure object with a Caption API and a way to capture that
    % in the conversion process - todo).
    \usepackage{caption}
    \DeclareCaptionFormat{nocaption}{}
    \captionsetup{format=nocaption,aboveskip=0pt,belowskip=0pt}

    \usepackage{float}
    \floatplacement{figure}{H} % forces figures to be placed at the correct location
    \usepackage{xcolor} % Allow colors to be defined
    \usepackage{enumerate} % Needed for markdown enumerations to work
    \usepackage{geometry} % Used to adjust the document margins
    \usepackage{amsmath} % Equations
    \usepackage{amssymb} % Equations
    \usepackage{textcomp} % defines textquotesingle
    % Hack from http://tex.stackexchange.com/a/47451/13684:
    \AtBeginDocument{%
        \def\PYZsq{\textquotesingle}% Upright quotes in Pygmentized code
    }
    \usepackage{upquote} % Upright quotes for verbatim code
    \usepackage{eurosym} % defines \euro
    \usepackage[mathletters]{ucs} % Extended unicode (utf-8) support
    \usepackage{fancyvrb} % verbatim replacement that allows latex
    \usepackage{grffile} % extends the file name processing of package graphics 
                         % to support a larger range
    \makeatletter % fix for old versions of grffile with XeLaTeX
    \@ifpackagelater{grffile}{2019/11/01}
    {
      % Do nothing on new versions
    }
    {
      \def\Gread@@xetex#1{%
        \IfFileExists{"\Gin@base".bb}%
        {\Gread@eps{\Gin@base.bb}}%
        {\Gread@@xetex@aux#1}%
      }
    }
    \makeatother
    \usepackage[Export]{adjustbox} % Used to constrain images to a maximum size
    \adjustboxset{max size={0.9\linewidth}{0.9\paperheight}}

    % The hyperref package gives us a pdf with properly built
    % internal navigation ('pdf bookmarks' for the table of contents,
    % internal cross-reference links, web links for URLs, etc.)
    \usepackage{hyperref}
    % The default LaTeX title has an obnoxious amount of whitespace. By default,
    % titling removes some of it. It also provides customization options.
    \usepackage{titling}
    \usepackage{longtable} % longtable support required by pandoc >1.10
    \usepackage{booktabs}  % table support for pandoc > 1.12.2
    \usepackage[inline]{enumitem} % IRkernel/repr support (it uses the enumerate* environment)
    \usepackage[normalem]{ulem} % ulem is needed to support strikethroughs (\sout)
                                % normalem makes italics be italics, not underlines
    \usepackage{mathrsfs}
    
\usepackage{fancyhdr}
\usepackage{lastpage}
\usepackage{cclicenses}

    % Colors for the hyperref package
    \definecolor{urlcolor}{rgb}{0,.145,.698}
    \definecolor{linkcolor}{rgb}{.71,0.21,0.01}
    \definecolor{citecolor}{rgb}{.12,.54,.11}

    % ANSI colors
    \definecolor{ansi-black}{HTML}{3E424D}
    \definecolor{ansi-black-intense}{HTML}{282C36}
    \definecolor{ansi-red}{HTML}{E75C58}
    \definecolor{ansi-red-intense}{HTML}{B22B31}
    \definecolor{ansi-green}{HTML}{00A250}
    \definecolor{ansi-green-intense}{HTML}{007427}
    \definecolor{ansi-yellow}{HTML}{DDB62B}
    \definecolor{ansi-yellow-intense}{HTML}{B27D12}
    \definecolor{ansi-blue}{HTML}{208FFB}
    \definecolor{ansi-blue-intense}{HTML}{0065CA}
    \definecolor{ansi-magenta}{HTML}{D160C4}
    \definecolor{ansi-magenta-intense}{HTML}{A03196}
    \definecolor{ansi-cyan}{HTML}{60C6C8}
    \definecolor{ansi-cyan-intense}{HTML}{258F8F}
    \definecolor{ansi-white}{HTML}{C5C1B4}
    \definecolor{ansi-white-intense}{HTML}{A1A6B2}
    \definecolor{ansi-default-inverse-fg}{HTML}{FFFFFF}
    \definecolor{ansi-default-inverse-bg}{HTML}{000000}

    % common color for the border for error outputs.
    \definecolor{outerrorbackground}{HTML}{FFDFDF}

    % commands and environments needed by pandoc snippets
    % extracted from the output of `pandoc -s`
    \providecommand{\tightlist}{%
      \setlength{\itemsep}{0pt}\setlength{\parskip}{0pt}}
    \DefineVerbatimEnvironment{Highlighting}{Verbatim}{commandchars=\\\{\}}
    % Add ',fontsize=\small' for more characters per line
    \newenvironment{Shaded}{}{}
    \newcommand{\KeywordTok}[1]{\textcolor[rgb]{0.00,0.44,0.13}{\textbf{{#1}}}}
    \newcommand{\DataTypeTok}[1]{\textcolor[rgb]{0.56,0.13,0.00}{{#1}}}
    \newcommand{\DecValTok}[1]{\textcolor[rgb]{0.25,0.63,0.44}{{#1}}}
    \newcommand{\BaseNTok}[1]{\textcolor[rgb]{0.25,0.63,0.44}{{#1}}}
    \newcommand{\FloatTok}[1]{\textcolor[rgb]{0.25,0.63,0.44}{{#1}}}
    \newcommand{\CharTok}[1]{\textcolor[rgb]{0.25,0.44,0.63}{{#1}}}
    \newcommand{\StringTok}[1]{\textcolor[rgb]{0.25,0.44,0.63}{{#1}}}
    \newcommand{\CommentTok}[1]{\textcolor[rgb]{0.38,0.63,0.69}{\textit{{#1}}}}
    \newcommand{\OtherTok}[1]{\textcolor[rgb]{0.00,0.44,0.13}{{#1}}}
    \newcommand{\AlertTok}[1]{\textcolor[rgb]{1.00,0.00,0.00}{\textbf{{#1}}}}
    \newcommand{\FunctionTok}[1]{\textcolor[rgb]{0.02,0.16,0.49}{{#1}}}
    \newcommand{\RegionMarkerTok}[1]{{#1}}
    \newcommand{\ErrorTok}[1]{\textcolor[rgb]{1.00,0.00,0.00}{\textbf{{#1}}}}
    \newcommand{\NormalTok}[1]{{#1}}
    
    % Additional commands for more recent versions of Pandoc
    \newcommand{\ConstantTok}[1]{\textcolor[rgb]{0.53,0.00,0.00}{{#1}}}
    \newcommand{\SpecialCharTok}[1]{\textcolor[rgb]{0.25,0.44,0.63}{{#1}}}
    \newcommand{\VerbatimStringTok}[1]{\textcolor[rgb]{0.25,0.44,0.63}{{#1}}}
    \newcommand{\SpecialStringTok}[1]{\textcolor[rgb]{0.73,0.40,0.53}{{#1}}}
    \newcommand{\ImportTok}[1]{{#1}}
    \newcommand{\DocumentationTok}[1]{\textcolor[rgb]{0.73,0.13,0.13}{\textit{{#1}}}}
    \newcommand{\AnnotationTok}[1]{\textcolor[rgb]{0.38,0.63,0.69}{\textbf{\textit{{#1}}}}}
    \newcommand{\CommentVarTok}[1]{\textcolor[rgb]{0.38,0.63,0.69}{\textbf{\textit{{#1}}}}}
    \newcommand{\VariableTok}[1]{\textcolor[rgb]{0.10,0.09,0.49}{{#1}}}
    \newcommand{\ControlFlowTok}[1]{\textcolor[rgb]{0.00,0.44,0.13}{\textbf{{#1}}}}
    \newcommand{\OperatorTok}[1]{\textcolor[rgb]{0.40,0.40,0.40}{{#1}}}
    \newcommand{\BuiltInTok}[1]{{#1}}
    \newcommand{\ExtensionTok}[1]{{#1}}
    \newcommand{\PreprocessorTok}[1]{\textcolor[rgb]{0.74,0.48,0.00}{{#1}}}
    \newcommand{\AttributeTok}[1]{\textcolor[rgb]{0.49,0.56,0.16}{{#1}}}
    \newcommand{\InformationTok}[1]{\textcolor[rgb]{0.38,0.63,0.69}{\textbf{\textit{{#1}}}}}
    \newcommand{\WarningTok}[1]{\textcolor[rgb]{0.38,0.63,0.69}{\textbf{\textit{{#1}}}}}
    
    % Slightly bigger margins than the latex defaults
    
    \geometry{verbose,tmargin=0.7in,bmargin=0.7in,lmargin=0.7in,rmargin=0.7in}
        
    % Define a nice break command that doesn't care if a line doesn't already
    % exist.
    \def\br{\hspace*{\fill} \\* }
    % Math Jax compatibility definitions
    \def\gt{>}
    \def\lt{<}
    \let\Oldtex\TeX
    \let\Oldlatex\LaTeX
    \renewcommand{\TeX}{\textrm{\Oldtex}}
    \renewcommand{\LaTeX}{\textrm{\Oldlatex}}
    % Document parameters
    % Document title
    \title{Opérateurs logiques}
      \date{Octobre 2021}  
	%\author{Yannick Chistel}
    
\makeatletter         
\renewcommand\maketitle[1]{
\hrule\medskip
{\raggedright % Note the extra {
\begin{center}
{\Huge \bfseries \sffamily #1 }\\[4ex] 
%{\Large  \@author}\\[2ex] 
%\@date\\[4ex]
\hrule \bigskip
\end{center}}} % Note the extra }
\makeatother    



\pagestyle{fancy}
\fancyhead{}
\renewcommand\headrulewidth{0pt}
\renewcommand\footrulewidth{1pt}
\fancyfoot[L]{YC}
\fancyfoot[C]{\thepage}
\fancyfoot[R]{\cc-\ccby-\ccnc}

% 
%\newtcolorbox{exemple}[2][]{
%    enhanced,
%    size=fbox,sharp corners,
%    colback=white,colframe=black,
%    colbacktitle=black,fonttitle=\bfseries,
%    attach boxed title to top left={yshift=-3mm,yshifttext=-3mm},
%    boxed title style={size=small,left=0pt,right=0pt,sharp corners},title=#2,#1}

\newtcolorbox{remarque}[2][]{colback=red!4!white,
colframe=red!64!black,fonttitle=\bfseries,
colbacktitle=red!64!black,enhanced,
attach boxed title to top left={xshift=4mm,yshift=-2mm},
title=#2,#1}  


\newtcolorbox{exemple}[2][]{colback=blue!4!white,
colframe=blue!64!green,fonttitle=\bfseries,
colbacktitle=blue!64!green,enhanced,
attach boxed title to top left={xshift=4mm,yshift=-2mm},
title=#2,#1}  
    
% Pygments definitions
\makeatletter
\def\PY@reset{\let\PY@it=\relax \let\PY@bf=\relax%
    \let\PY@ul=\relax \let\PY@tc=\relax%
    \let\PY@bc=\relax \let\PY@ff=\relax}
\def\PY@tok#1{\csname PY@tok@#1\endcsname}
\def\PY@toks#1+{\ifx\relax#1\empty\else%
    \PY@tok{#1}\expandafter\PY@toks\fi}
\def\PY@do#1{\PY@bc{\PY@tc{\PY@ul{%
    \PY@it{\PY@bf{\PY@ff{#1}}}}}}}
\def\PY#1#2{\PY@reset\PY@toks#1+\relax+\PY@do{#2}}

\@namedef{PY@tok@w}{\def\PY@tc##1{\textcolor[rgb]{0.73,0.73,0.73}{##1}}}
\@namedef{PY@tok@c}{\let\PY@it=\textit\def\PY@tc##1{\textcolor[rgb]{0.25,0.50,0.50}{##1}}}
\@namedef{PY@tok@cp}{\def\PY@tc##1{\textcolor[rgb]{0.74,0.48,0.00}{##1}}}
\@namedef{PY@tok@k}{\let\PY@bf=\textbf\def\PY@tc##1{\textcolor[rgb]{0.00,0.50,0.00}{##1}}}
\@namedef{PY@tok@kp}{\def\PY@tc##1{\textcolor[rgb]{0.00,0.50,0.00}{##1}}}
\@namedef{PY@tok@kt}{\def\PY@tc##1{\textcolor[rgb]{0.69,0.00,0.25}{##1}}}
\@namedef{PY@tok@o}{\def\PY@tc##1{\textcolor[rgb]{0.40,0.40,0.40}{##1}}}
\@namedef{PY@tok@ow}{\let\PY@bf=\textbf\def\PY@tc##1{\textcolor[rgb]{0.67,0.13,1.00}{##1}}}
\@namedef{PY@tok@nb}{\def\PY@tc##1{\textcolor[rgb]{0.00,0.50,0.00}{##1}}}
\@namedef{PY@tok@nf}{\def\PY@tc##1{\textcolor[rgb]{0.00,0.00,1.00}{##1}}}
\@namedef{PY@tok@nc}{\let\PY@bf=\textbf\def\PY@tc##1{\textcolor[rgb]{0.00,0.00,1.00}{##1}}}
\@namedef{PY@tok@nn}{\let\PY@bf=\textbf\def\PY@tc##1{\textcolor[rgb]{0.00,0.00,1.00}{##1}}}
\@namedef{PY@tok@ne}{\let\PY@bf=\textbf\def\PY@tc##1{\textcolor[rgb]{0.82,0.25,0.23}{##1}}}
\@namedef{PY@tok@nv}{\def\PY@tc##1{\textcolor[rgb]{0.10,0.09,0.49}{##1}}}
\@namedef{PY@tok@no}{\def\PY@tc##1{\textcolor[rgb]{0.53,0.00,0.00}{##1}}}
\@namedef{PY@tok@nl}{\def\PY@tc##1{\textcolor[rgb]{0.63,0.63,0.00}{##1}}}
\@namedef{PY@tok@ni}{\let\PY@bf=\textbf\def\PY@tc##1{\textcolor[rgb]{0.60,0.60,0.60}{##1}}}
\@namedef{PY@tok@na}{\def\PY@tc##1{\textcolor[rgb]{0.49,0.56,0.16}{##1}}}
\@namedef{PY@tok@nt}{\let\PY@bf=\textbf\def\PY@tc##1{\textcolor[rgb]{0.00,0.50,0.00}{##1}}}
\@namedef{PY@tok@nd}{\def\PY@tc##1{\textcolor[rgb]{0.67,0.13,1.00}{##1}}}
\@namedef{PY@tok@s}{\def\PY@tc##1{\textcolor[rgb]{0.73,0.13,0.13}{##1}}}
\@namedef{PY@tok@sd}{\let\PY@it=\textit\def\PY@tc##1{\textcolor[rgb]{0.73,0.13,0.13}{##1}}}
\@namedef{PY@tok@si}{\let\PY@bf=\textbf\def\PY@tc##1{\textcolor[rgb]{0.73,0.40,0.53}{##1}}}
\@namedef{PY@tok@se}{\let\PY@bf=\textbf\def\PY@tc##1{\textcolor[rgb]{0.73,0.40,0.13}{##1}}}
\@namedef{PY@tok@sr}{\def\PY@tc##1{\textcolor[rgb]{0.73,0.40,0.53}{##1}}}
\@namedef{PY@tok@ss}{\def\PY@tc##1{\textcolor[rgb]{0.10,0.09,0.49}{##1}}}
\@namedef{PY@tok@sx}{\def\PY@tc##1{\textcolor[rgb]{0.00,0.50,0.00}{##1}}}
\@namedef{PY@tok@m}{\def\PY@tc##1{\textcolor[rgb]{0.40,0.40,0.40}{##1}}}
\@namedef{PY@tok@gh}{\let\PY@bf=\textbf\def\PY@tc##1{\textcolor[rgb]{0.00,0.00,0.50}{##1}}}
\@namedef{PY@tok@gu}{\let\PY@bf=\textbf\def\PY@tc##1{\textcolor[rgb]{0.50,0.00,0.50}{##1}}}
\@namedef{PY@tok@gd}{\def\PY@tc##1{\textcolor[rgb]{0.63,0.00,0.00}{##1}}}
\@namedef{PY@tok@gi}{\def\PY@tc##1{\textcolor[rgb]{0.00,0.63,0.00}{##1}}}
\@namedef{PY@tok@gr}{\def\PY@tc##1{\textcolor[rgb]{1.00,0.00,0.00}{##1}}}
\@namedef{PY@tok@ge}{\let\PY@it=\textit}
\@namedef{PY@tok@gs}{\let\PY@bf=\textbf}
\@namedef{PY@tok@gp}{\let\PY@bf=\textbf\def\PY@tc##1{\textcolor[rgb]{0.00,0.00,0.50}{##1}}}
\@namedef{PY@tok@go}{\def\PY@tc##1{\textcolor[rgb]{0.53,0.53,0.53}{##1}}}
\@namedef{PY@tok@gt}{\def\PY@tc##1{\textcolor[rgb]{0.00,0.27,0.87}{##1}}}
\@namedef{PY@tok@err}{\def\PY@bc##1{{\setlength{\fboxsep}{\string -\fboxrule}\fcolorbox[rgb]{1.00,0.00,0.00}{1,1,1}{\strut ##1}}}}
\@namedef{PY@tok@kc}{\let\PY@bf=\textbf\def\PY@tc##1{\textcolor[rgb]{0.00,0.50,0.00}{##1}}}
\@namedef{PY@tok@kd}{\let\PY@bf=\textbf\def\PY@tc##1{\textcolor[rgb]{0.00,0.50,0.00}{##1}}}
\@namedef{PY@tok@kn}{\let\PY@bf=\textbf\def\PY@tc##1{\textcolor[rgb]{0.00,0.50,0.00}{##1}}}
\@namedef{PY@tok@kr}{\let\PY@bf=\textbf\def\PY@tc##1{\textcolor[rgb]{0.00,0.50,0.00}{##1}}}
\@namedef{PY@tok@bp}{\def\PY@tc##1{\textcolor[rgb]{0.00,0.50,0.00}{##1}}}
\@namedef{PY@tok@fm}{\def\PY@tc##1{\textcolor[rgb]{0.00,0.00,1.00}{##1}}}
\@namedef{PY@tok@vc}{\def\PY@tc##1{\textcolor[rgb]{0.10,0.09,0.49}{##1}}}
\@namedef{PY@tok@vg}{\def\PY@tc##1{\textcolor[rgb]{0.10,0.09,0.49}{##1}}}
\@namedef{PY@tok@vi}{\def\PY@tc##1{\textcolor[rgb]{0.10,0.09,0.49}{##1}}}
\@namedef{PY@tok@vm}{\def\PY@tc##1{\textcolor[rgb]{0.10,0.09,0.49}{##1}}}
\@namedef{PY@tok@sa}{\def\PY@tc##1{\textcolor[rgb]{0.73,0.13,0.13}{##1}}}
\@namedef{PY@tok@sb}{\def\PY@tc##1{\textcolor[rgb]{0.73,0.13,0.13}{##1}}}
\@namedef{PY@tok@sc}{\def\PY@tc##1{\textcolor[rgb]{0.73,0.13,0.13}{##1}}}
\@namedef{PY@tok@dl}{\def\PY@tc##1{\textcolor[rgb]{0.73,0.13,0.13}{##1}}}
\@namedef{PY@tok@s2}{\def\PY@tc##1{\textcolor[rgb]{0.73,0.13,0.13}{##1}}}
\@namedef{PY@tok@sh}{\def\PY@tc##1{\textcolor[rgb]{0.73,0.13,0.13}{##1}}}
\@namedef{PY@tok@s1}{\def\PY@tc##1{\textcolor[rgb]{0.73,0.13,0.13}{##1}}}
\@namedef{PY@tok@mb}{\def\PY@tc##1{\textcolor[rgb]{0.40,0.40,0.40}{##1}}}
\@namedef{PY@tok@mf}{\def\PY@tc##1{\textcolor[rgb]{0.40,0.40,0.40}{##1}}}
\@namedef{PY@tok@mh}{\def\PY@tc##1{\textcolor[rgb]{0.40,0.40,0.40}{##1}}}
\@namedef{PY@tok@mi}{\def\PY@tc##1{\textcolor[rgb]{0.40,0.40,0.40}{##1}}}
\@namedef{PY@tok@il}{\def\PY@tc##1{\textcolor[rgb]{0.40,0.40,0.40}{##1}}}
\@namedef{PY@tok@mo}{\def\PY@tc##1{\textcolor[rgb]{0.40,0.40,0.40}{##1}}}
\@namedef{PY@tok@ch}{\let\PY@it=\textit\def\PY@tc##1{\textcolor[rgb]{0.25,0.50,0.50}{##1}}}
\@namedef{PY@tok@cm}{\let\PY@it=\textit\def\PY@tc##1{\textcolor[rgb]{0.25,0.50,0.50}{##1}}}
\@namedef{PY@tok@cpf}{\let\PY@it=\textit\def\PY@tc##1{\textcolor[rgb]{0.25,0.50,0.50}{##1}}}
\@namedef{PY@tok@c1}{\let\PY@it=\textit\def\PY@tc##1{\textcolor[rgb]{0.25,0.50,0.50}{##1}}}
\@namedef{PY@tok@cs}{\let\PY@it=\textit\def\PY@tc##1{\textcolor[rgb]{0.25,0.50,0.50}{##1}}}

\def\PYZbs{\char`\\}
\def\PYZus{\char`\_}
\def\PYZob{\char`\{}
\def\PYZcb{\char`\}}
\def\PYZca{\char`\^}
\def\PYZam{\char`\&}
\def\PYZlt{\char`\<}
\def\PYZgt{\char`\>}
\def\PYZsh{\char`\#}
\def\PYZpc{\char`\%}
\def\PYZdl{\char`\$}
\def\PYZhy{\char`\-}
\def\PYZsq{\char`\'}
\def\PYZdq{\char`\"}
\def\PYZti{\char`\~}
% for compatibility with earlier versions
\def\PYZat{@}
\def\PYZlb{[}
\def\PYZrb{]}
\makeatother


    % For linebreaks inside Verbatim environment from package fancyvrb. 
    \makeatletter
        \newbox\Wrappedcontinuationbox 
        \newbox\Wrappedvisiblespacebox 
        \newcommand*\Wrappedvisiblespace {\textcolor{red}{\textvisiblespace}} 
        \newcommand*\Wrappedcontinuationsymbol {\textcolor{red}{\llap{\tiny$\m@th\hookrightarrow$}}} 
        \newcommand*\Wrappedcontinuationindent {3ex } 
        \newcommand*\Wrappedafterbreak {\kern\Wrappedcontinuationindent\copy\Wrappedcontinuationbox} 
        % Take advantage of the already applied Pygments mark-up to insert 
        % potential linebreaks for TeX processing. 
        %        {, <, #, %, $, ' and ": go to next line. 
        %        _, }, ^, &, >, - and ~: stay at end of broken line. 
        % Use of \textquotesingle for straight quote. 
        \newcommand*\Wrappedbreaksatspecials {% 
            \def\PYGZus{\discretionary{\char`\_}{\Wrappedafterbreak}{\char`\_}}% 
            \def\PYGZob{\discretionary{}{\Wrappedafterbreak\char`\{}{\char`\{}}% 
            \def\PYGZcb{\discretionary{\char`\}}{\Wrappedafterbreak}{\char`\}}}% 
            \def\PYGZca{\discretionary{\char`\^}{\Wrappedafterbreak}{\char`\^}}% 
            \def\PYGZam{\discretionary{\char`\&}{\Wrappedafterbreak}{\char`\&}}% 
            \def\PYGZlt{\discretionary{}{\Wrappedafterbreak\char`\<}{\char`\<}}% 
            \def\PYGZgt{\discretionary{\char`\>}{\Wrappedafterbreak}{\char`\>}}% 
            \def\PYGZsh{\discretionary{}{\Wrappedafterbreak\char`\#}{\char`\#}}% 
            \def\PYGZpc{\discretionary{}{\Wrappedafterbreak\char`\%}{\char`\%}}% 
            \def\PYGZdl{\discretionary{}{\Wrappedafterbreak\char`\$}{\char`\$}}% 
            \def\PYGZhy{\discretionary{\char`\-}{\Wrappedafterbreak}{\char`\-}}% 
            \def\PYGZsq{\discretionary{}{\Wrappedafterbreak\textquotesingle}{\textquotesingle}}% 
            \def\PYGZdq{\discretionary{}{\Wrappedafterbreak\char`\"}{\char`\"}}% 
            \def\PYGZti{\discretionary{\char`\~}{\Wrappedafterbreak}{\char`\~}}% 
        } 
        % Some characters . , ; ? ! / are not pygmentized. 
        % This macro makes them "active" and they will insert potential linebreaks 
        \newcommand*\Wrappedbreaksatpunct {% 
            \lccode`\~`\.\lowercase{\def~}{\discretionary{\hbox{\char`\.}}{\Wrappedafterbreak}{\hbox{\char`\.}}}% 
            \lccode`\~`\,\lowercase{\def~}{\discretionary{\hbox{\char`\,}}{\Wrappedafterbreak}{\hbox{\char`\,}}}% 
            \lccode`\~`\;\lowercase{\def~}{\discretionary{\hbox{\char`\;}}{\Wrappedafterbreak}{\hbox{\char`\;}}}% 
            \lccode`\~`\:\lowercase{\def~}{\discretionary{\hbox{\char`\:}}{\Wrappedafterbreak}{\hbox{\char`\:}}}% 
            \lccode`\~`\?\lowercase{\def~}{\discretionary{\hbox{\char`\?}}{\Wrappedafterbreak}{\hbox{\char`\?}}}% 
            \lccode`\~`\!\lowercase{\def~}{\discretionary{\hbox{\char`\!}}{\Wrappedafterbreak}{\hbox{\char`\!}}}% 
            \lccode`\~`\/\lowercase{\def~}{\discretionary{\hbox{\char`\/}}{\Wrappedafterbreak}{\hbox{\char`\/}}}% 
            \catcode`\.\active
            \catcode`\,\active 
            \catcode`\;\active
            \catcode`\:\active
            \catcode`\?\active
            \catcode`\!\active
            \catcode`\/\active 
            \lccode`\~`\~ 	
        }
    \makeatother

    \let\OriginalVerbatim=\Verbatim
    \makeatletter
    \renewcommand{\Verbatim}[1][1]{%
        %\parskip\z@skip
        \sbox\Wrappedcontinuationbox {\Wrappedcontinuationsymbol}%
        \sbox\Wrappedvisiblespacebox {\FV@SetupFont\Wrappedvisiblespace}%
        \def\FancyVerbFormatLine ##1{\hsize\linewidth
            \vtop{\raggedright\hyphenpenalty\z@\exhyphenpenalty\z@
                \doublehyphendemerits\z@\finalhyphendemerits\z@
                \strut ##1\strut}%
        }%
        % If the linebreak is at a space, the latter will be displayed as visible
        % space at end of first line, and a continuation symbol starts next line.
        % Stretch/shrink are however usually zero for typewriter font.
        \def\FV@Space {%
            \nobreak\hskip\z@ plus\fontdimen3\font minus\fontdimen4\font
            \discretionary{\copy\Wrappedvisiblespacebox}{\Wrappedafterbreak}
            {\kern\fontdimen2\font}%
        }%
        
        % Allow breaks at special characters using \PYG... macros.
        \Wrappedbreaksatspecials
        % Breaks at punctuation characters . , ; ? ! and / need catcode=\active 	
        \OriginalVerbatim[#1,codes*=\Wrappedbreaksatpunct]%
    }
    \makeatother

    % Exact colors from NB
    \definecolor{incolor}{HTML}{303F9F}
    \definecolor{outcolor}{HTML}{D84315}
    \definecolor{cellborder}{HTML}{CFCFCF}
    \definecolor{cellbackground}{HTML}{F7F7F7}
    
    % prompt
    \makeatletter
    \newcommand{\boxspacing}{\kern\kvtcb@left@rule\kern\kvtcb@boxsep}
    \makeatother
    \newcommand{\prompt}[4]{
        {\ttfamily\llap{{\color{#2}[#3]:\hspace{3pt}#4}}\vspace{-\baselineskip}}
    }
    

    
    % Prevent overflowing lines due to hard-to-break entities
    \sloppy 
    % Setup hyperref package
    \hypersetup{
      breaklinks=true,  % so long urls are correctly broken across lines
      colorlinks=true,
      urlcolor=urlcolor,
      linkcolor=linkcolor,
      citecolor=citecolor,
      }

    


\begin{document}
    
    \maketitle{Programmation orientée objet (POO)}
    


    
%    \hypertarget{programmation-orientuxe9e-objet-poo}{%
%\section{Programmation orientée objet
%(POO)}\label{programmation-orientuxe9e-objet-poo}}

    \hypertarget{introduction}{%
\section{Introduction}\label{introduction}}

Le paradigme de \textbf{programmation orientée objet} s'appuie sur les
notions suivantes :

\begin{itemize}
\tightlist
\item
  Un \textbf{objet} est un modèle, un moule qui va permettre de créer
  des représentations de cet objet. Chaque représentation est une
  \textbf{instance};
\item
  Chaque instance d'objet a des valeurs définies à la construction de
  l'objet. Ces valeurs sont les \textbf{attributs} de l'objet;
\item
  Un objet possède différentes fonctionnalités. Toutes les instances
  bénéficieront de ces fonctions. On les appelle des \textbf{méthodes}.
\end{itemize}

    \hypertarget{objet-en-python}{%
\subsection{Objet en Python}\label{objet-en-python}}

En python, un objet est défini par le mot clef \textbf{class} suivi du
nom de l'objet et des 2 points.

\begin{Shaded}
\begin{Highlighting}[]
\KeywordTok{class}\NormalTok{ objet:}
    \CommentTok{\# attributs et méthodes de l\textquotesingle{}objet}
\end{Highlighting}
\end{Shaded}

Une \textbf{classe} est une structure de donnée qui permet de créer des
objets.\\
Une \textbf{classe} permet de définir les attributs d'un objet et les
méthodes qui lui sont propres. Tout ce qui est défini dans la classe
doit être \textbf{indenté}.

La création d'un objet se fera par une affectation avec la classe
définissant l'objet. On dit qu'on \textbf{instancie} un objet.

    \hypertarget{exemple}{%
\subsubsection*{Exemple}\label{exemple}}

Imaginons que l'on souhaite créer un objet en python représentant une
voiture. On peut définir les attributs et les méthodes de notre objet
comme suit:

\begin{itemize}
\tightlist
\item
  Les attributs sont des caractéristiques ou des propriétés de la
  voiture: le nombre de roues, la marque, le modèle, le nombre de
  portes, la couleur, etc.
\item
  les méthodes représentent des fonctionnalités ou des actions: avancer,
  accélérer, démarrer, ralentir, etc.
\end{itemize}

Pour créer nos objets voitures, on définit la classe automobile:

\begin{Shaded}
\begin{Highlighting}[]
\KeywordTok{class}\NormalTok{ automobile:}
    \ControlFlowTok{pass} \CommentTok{\# instruction qui ne fait rien mais évite une erreur dans l\textquotesingle{}interpréteur}
\end{Highlighting}
\end{Shaded}

Dans l'interpréteur ou le notebook, on peut créer différentes voitures:

\begin{Shaded}
\begin{Highlighting}[]
\NormalTok{clio}\OperatorTok{=}\NormalTok{automobile()}
\NormalTok{polo}\OperatorTok{=}\NormalTok{automobile()}
\end{Highlighting}
\end{Shaded}

\hypertarget{remarque}{%
\subsubsection*{Remarque}\label{remarque}}

Nous avons deux instances d'objets construites avec la classe automobile
: \textbf{polo} et \textbf{clio}.\\
On fera souvent le raccourci que \textbf{clio} et \textbf{polo} sont
deux objets de la classe automobile.

    \hypertarget{attributs-et-muxe9thodes-dun-objet}{%
\section{Attributs et méthodes d'un
objet}\label{attributs-et-muxe9thodes-dun-objet}}

Nous avons créé deux objets mais ils n'ont ni attributs ni méthodes
(puisque la classe est vide). On peut ajouter des attributs et des
méthodes à un objet dans l'interpréteur (mais ce n'est pas la bonne
méthode).

Comme pour tout objet, l'accès aux attributs et aux méthodes se fait
avec la syntaxe suivante:

\begin{Shaded}
\begin{Highlighting}[]
\NormalTok{objet.attribut }\CommentTok{\# appel d\textquotesingle{}un attribut de l\textquotesingle{}objet}
\NormalTok{objet.méthode() }\CommentTok{\# appel d\textquotesingle{}une méthode de l\textquotesingle{}objet, les parenthèses rappellent que c\textquotesingle{}est une fonction.}
\end{Highlighting}
\end{Shaded}

\hypertarget{les-attributs-dun-objet}{%
\subsection{Les attributs d'un objet}\label{les-attributs-dun-objet}}

Les attributs d'un objet permettent de stocker des valeurs pour notre
objet. Ces attributs sont accessibles et peuvent être modifiés. Dans
certains langages, l'accès aux attributs est protégé et nécessite des
fonctions pour accéder et modifier l'attribut. En python, l'accès est
libre par défaut mais on peut le protéger avec une fonction.

Reprenons notre exemple de voiture. On peut définir comme attribut le
nombre de roues et le nombre de portes.

\begin{Shaded}
\begin{Highlighting}[]
\CommentTok{\# la clio a 4 roues et trois portes}
\NormalTok{clio.roues }\OperatorTok{=} \DecValTok{4}
\NormalTok{clio.portes }\OperatorTok{=} \DecValTok{3}

\CommentTok{\# la polo a 4 roues et 5 portes}
\NormalTok{polo.roues }\OperatorTok{=} \DecValTok{4}
\NormalTok{polo.portes }\OperatorTok{=} \DecValTok{5}
\NormalTok{polo.carburant }\OperatorTok{=}\NormalTok{ diesel}
\end{Highlighting}
\end{Shaded}

On a défini deux attributs communs pour chaque objet. Il est possible de
définir un attribut pour un objet et pas pour l'autre. Bien que ce soit
possible, il vaut mieux éviter de le faire et plutôt créer des objets
uniformes avec les mêmes attributs pour éviter des erreurs.

\hypertarget{les-muxe9thodes-dun-objet}{%
\subsection{Les méthodes d'un objet}\label{les-muxe9thodes-dun-objet}}

Les méthodes sont des fonctions propres aux objets, ce qui implique que
la fonction ne peut être appliquée qu'à l'objet. Comme l'attribut, la
méthode est placée après le nom de l'objet séparée par un point :
\texttt{objet.méthode()}.

Une \textbf{méthode} renvoie:

\begin{itemize}
\tightlist
\item
  une valeur dont le type est un nombre (int, float), un bouléen (bool),
  une chaine de caractère (str) ou un type construit comme la liste
  (list) ou le dictionnaire (dict);
\item
  un autre objet (class);
\item
  rien ou None tout en agissant sur un attribut : modifier, créer;
\item
  un affichage.
\end{itemize}

En python, une méthode d'objet est définie dans la classe. Pour faire
référence à l'objet, nous devons utiliser le mot clef \textbf{self} qui
désignera l'objet. Ce mot clef \textbf{self} sera passé en paramètre et
précisé à chaque fois que l'objet sera référencé.

Reprenons l'exemple de l'objet automobile et créons une méthode pour
savoir si la voiture avance.

\hypertarget{exemple}{%
\subsubsection*{Exemple}\label{exemple}}

\begin{Shaded}
\begin{Highlighting}[]
\KeywordTok{class}\NormalTok{ automobile:}
    \CommentTok{\# attributs}
\NormalTok{    roues}\OperatorTok{=}\DecValTok{4}
\NormalTok{    portes}\OperatorTok{=}\DecValTok{3}
\NormalTok{    vitesse}\OperatorTok{=}\DecValTok{0}
    
    \CommentTok{\# méthode}
    \KeywordTok{def}\NormalTok{ avancer(}\VariableTok{self}\NormalTok{):}
        \ControlFlowTok{if} \VariableTok{self}\NormalTok{.vitesse }\OperatorTok{\textgreater{}} \DecValTok{0}\NormalTok{:}
            \ControlFlowTok{return} \VariableTok{True}
        \ControlFlowTok{else}\NormalTok{:}
            \ControlFlowTok{return} \VariableTok{False}

\CommentTok{\# premier objet : polo}
\NormalTok{polo}\OperatorTok{=}\NormalTok{automobile()}
\BuiltInTok{print}\NormalTok{(}\StringTok{"la polo a }\SpecialCharTok{\%s}\StringTok{ portes"} \OperatorTok{\%}\NormalTok{ polo.portes)}

\ControlFlowTok{if}\NormalTok{ polo.avancer():}
    \BuiltInTok{print}\NormalTok{(}\StringTok{"La polo avance."}\NormalTok{)}
\ControlFlowTok{else}\NormalTok{:}
    \BuiltInTok{print}\NormalTok{(}\StringTok{"La polo est à l\textquotesingle{}arrêt"}\NormalTok{)}

\CommentTok{\# second objet : clio}
\NormalTok{clio}\OperatorTok{=}\NormalTok{automobile()}
\CommentTok{\# on modifie la vitesse de la clio}
\NormalTok{clio.vitesse}\OperatorTok{=}\DecValTok{50}

\ControlFlowTok{if}\NormalTok{ clio.avancer():}
    \BuiltInTok{print}\NormalTok{(}\StringTok{"La clio avance."}\NormalTok{)}
\ControlFlowTok{else}\NormalTok{:}
    \BuiltInTok{print}\NormalTok{(}\StringTok{"La clio est à l\textquotesingle{}arrêt"}\NormalTok{)}
\end{Highlighting}
\end{Shaded}

Si on exécute ce programme, on obtient les affichages:

\begin{verbatim}
la polo a 3 portes.
la polo est à l'arrêt.
la clio avance.
\end{verbatim}

\hypertarget{remarque}{%
\subsubsection*{Remarque}\label{remarque}}

Ainsi définie, la classe impose un nombre de portes égal à 3 quel que
soit l'objet créé. Donc la polo et la clio ont trois portes. Dans ce cas
on parle \textbf{d'attribut de classe}.

Il est possible de modifier la valeur par une affectation :
\texttt{clio.portes\ =\ 5} après la création de l'objet mais cela n'est
pas très optimisé !

%    \hypertarget{exercice}{%
%\subsubsection*{Exercice}\label{exercice}}

\begin{exemple}{Exercice}
Créer une cellule de code Python et copier coller le code de l'exemple précédent sur la classe automobile.

\begin{enumerate}
\def\labelenumi{\arabic{enumi}.}
\tightlist
\item
  Modifier ce programme pour créer un nouvel objet représentant une
  voiture du modèle \emph{mini} avec 3 portes.
\item
  La polo a 5 portes et 5 roues (roue de secours). Modifier le programme
  en conséquence.
\item
  La mini roule à 70 km/h. Modifier le code.
\item
  Modifier l'affichage pour chaque véhicule en donnant le nombre de
  roues, le nombre de portes et son statut (arrêt ou mobile) en
  indiquant sa vitesse.
\end{enumerate}

\textbf{En Plus:} \emph{est-il possible d'optimiser le code pour éviter
la répétition des instructions ?}
\end{exemple}



\begin{exemple}{Exercice}
%    \hypertarget{exercice}{%
%\subsubsection*{Exercice}\label{exercice}}

On reprend le code précédent que l'on continue de modifier.

\begin{enumerate}
\def\labelenumi{\arabic{enumi}.}
\tightlist
\item
  Ajouter à la classe un nouvel attribut indiquant le nombre de places
  disponibles. La valeur par défaut sera égale à 4.
\item
  Ajouter la méthode accelerer à la classe. Cette méthode augmente la
  vitesse de l'objet. À chaque appel de la méthode, la vitesse augmente
  de 10.
\item
  La polo, initialement à l'arrêt, roule maintenant à 90 km/h. Écrire un
  code qui modifie la vitesse de la polo en utilisant la méthode
  accelerer.
\item
  Écrire une méthode ralentir qui diminue la vitesse par palier de 10.
\item
  Écrire une méthode arreter qui modifie la vitesse du véhicule à 0.
\end{enumerate}

\textbf{En Plus:} \emph{est-il possible d'utiliser un paramètre à la
méthode accelerer pour indiquer la valeur à ajouter à la vitesse ? De
même avec la méthode ralentir.}
\end{exemple}





    \hypertarget{le-constructeur-muxe9thode-pour-duxe9finir-les-attributs}{%
\section{Le constructeur : méthode pour définir les
attributs}\label{le-constructeur-muxe9thode-pour-duxe9finir-les-attributs}}

En python, il existe une méthode \textbf{\_\_init\_\_} qui permet de
définir les attributs à la création de l'objet. Cette méthode est un
\textbf{constructeur} d'objet qui initialise les attributs.

Comme toute méthode, elle aura un paramètre \textbf{self} et chaque
attribut sera préfixé par le mot \textbf{self}.

\hypertarget{exemple}{%
\subsubsection*{Exemple}\label{exemple}}

On initialise les attributs de la classe automobile avec ce constructeur
\textbf{\_\_init\_\_} :

\begin{Shaded}
\begin{Highlighting}[]
\KeywordTok{class}\NormalTok{ automobile:}
    
    \CommentTok{\# constructeur de l\textquotesingle{}objet}
    \KeywordTok{def} \FunctionTok{\_\_init\_\_}\NormalTok{(}\VariableTok{self}\NormalTok{):}
        \VariableTok{self}\NormalTok{.roues}\OperatorTok{=}\DecValTok{4}
        \VariableTok{self}\NormalTok{.portes}\OperatorTok{=}\DecValTok{3}
        \VariableTok{self}\NormalTok{.vitesse}\OperatorTok{=}\DecValTok{0}
    
    \CommentTok{\# les autres méthode de l\textquotesingle{}objet}
    \KeywordTok{def}\NormalTok{ avancer(}\VariableTok{self}\NormalTok{):}
        \ControlFlowTok{if} \VariableTok{self}\NormalTok{.vitesse }\OperatorTok{\textgreater{}} \DecValTok{0}\NormalTok{:}
            \ControlFlowTok{return} \VariableTok{True}
        \ControlFlowTok{else}\NormalTok{:}
            \ControlFlowTok{return} \VariableTok{False}

\CommentTok{\# premier objet : polo}
\NormalTok{polo}\OperatorTok{=}\NormalTok{automobile()}
\BuiltInTok{print}\NormalTok{(}\StringTok{"la polo a }\SpecialCharTok{\%s}\StringTok{ roues et }\SpecialCharTok{\%s}\StringTok{ portes."} \OperatorTok{\%}\NormalTok{ (polo.roues,polo.portes))}
\end{Highlighting}
\end{Shaded}

Le reste du code ne change pas ! Quel est alors le véritable intérêt ?

Les fonctions acceptent des \textbf{paramètres}. Cela permet donc de
passer, au moment de l'appel, des valeurs en argument pour modifier les
attributs de chaque objet.

\hypertarget{exemple-1}{%
\subsubsection*{Exemple}\label{exemple-1}}

En définissant des paramètres pour les différents attributs, on peut
créer des objets avec des valeurs particulières passées comme arguments:

\begin{Shaded}
\begin{Highlighting}[]
\KeywordTok{class}\NormalTok{ automobile:}
    
    \CommentTok{\# constructeur de l\textquotesingle{}objet}
    \KeywordTok{def} \FunctionTok{\_\_init\_\_}\NormalTok{(}\VariableTok{self}\NormalTok{,r,p,v}\OperatorTok{=}\DecValTok{0}\NormalTok{):}
        \VariableTok{self}\NormalTok{.roues }\OperatorTok{=}\NormalTok{ r }\CommentTok{\# r est le nombre de roues}
        \VariableTok{self}\NormalTok{.portes }\OperatorTok{=}\NormalTok{ p }\CommentTok{\# p est le nombre de portes}
        \VariableTok{self}\NormalTok{.vitesse }\OperatorTok{=}\NormalTok{ v }\CommentTok{\# v est la vitesse initiale par défaut égale à 0}
    
    \CommentTok{\# méthode de l\textquotesingle{}objet}
    \KeywordTok{def}\NormalTok{ avancer(}\VariableTok{self}\NormalTok{):}
        \ControlFlowTok{if} \VariableTok{self}\NormalTok{.vitesse }\OperatorTok{\textgreater{}} \DecValTok{0}\NormalTok{:}
            \ControlFlowTok{return} \VariableTok{True}
        \ControlFlowTok{else}\NormalTok{:}
            \ControlFlowTok{return} \VariableTok{False}

\CommentTok{\# premier objet : polo}
\NormalTok{polo}\OperatorTok{=}\NormalTok{automobile(}\DecValTok{4}\NormalTok{,}\DecValTok{5}\NormalTok{)}

\BuiltInTok{print}\NormalTok{(}\StringTok{"La polo a }\SpecialCharTok{\%s}\StringTok{ roues et }\SpecialCharTok{\%s}\StringTok{ portes"} \OperatorTok{\%}\NormalTok{ (polo.roues,polo.portes))}
\ControlFlowTok{if}\NormalTok{ polo.avancer():}
    \BuiltInTok{print}\NormalTok{(}\StringTok{"La polo avance à }\SpecialCharTok{\%s}\StringTok{ km/h."} \OperatorTok{\%}\NormalTok{ polo.vitesse)}
\ControlFlowTok{else}\NormalTok{:}
    \BuiltInTok{print}\NormalTok{(}\StringTok{"La polo est à l\textquotesingle{}arrêt."}\NormalTok{)}


\CommentTok{\# second objet : clio}
\NormalTok{clio}\OperatorTok{=}\NormalTok{automobile(}\DecValTok{4}\NormalTok{,}\DecValTok{3}\NormalTok{,}\DecValTok{40}\NormalTok{)}

\BuiltInTok{print}\NormalTok{(}\StringTok{"La clio a }\SpecialCharTok{\%s}\StringTok{ roues et }\SpecialCharTok{\%s}\StringTok{ portes"} \OperatorTok{\%}\NormalTok{ (clio.roues,clio.portes))}
\BuiltInTok{print}\NormalTok{(}\StringTok{"la clio a }\SpecialCharTok{\%s}\StringTok{ portes"} \OperatorTok{\%}\NormalTok{ clio.portes)}
\ControlFlowTok{if}\NormalTok{ clio.avancer():}
    \BuiltInTok{print}\NormalTok{(}\StringTok{"La clio avance à }\SpecialCharTok{\%s}\StringTok{ km/h."} \OperatorTok{\%}\NormalTok{ clio.vitesse)}
\ControlFlowTok{else}\NormalTok{:}
    \BuiltInTok{print}\NormalTok{(}\StringTok{"La clio est à l\textquotesingle{}arrêt."}\NormalTok{)}
\end{Highlighting}
\end{Shaded}

%\hypertarget{exercice}{%
%\subsubsection*{Exercice}\label{exercice}}


\begin{exemple}{Exercice}
On reprend le code de l'exercice précédent.

\begin{enumerate}
\def\labelenumi{\arabic{enumi}.}
\tightlist
\item
  Ajouter le constructeur à votre classe automobile (attention au nombre
  d'attributs).
\item
  Créer avec ce constructeur les trois automobiles \emph{polo},
  \emph{clio} et \emph{mini}.
\item
  Afficher les attributs de chaque voiture.
\item
  Ajouter les attributs nom et couleur à la classe automobile.
\item
  Recréez vos trois voitures de couleurs différentes en utilisant comme
  nom de variable \emph{a1}, \emph{a2} et \emph{a3}
\end{enumerate}

\textbf{En Plus:} \emph{est-il possible d'optimiser le code pour
l'affichage ?}
\end{exemple}
    
    
    
\end{document}
